\documentclass[11pt]{beamer}    % Presentacion
\usepackage{graphicx}


\usetheme{metropolis}           % Usar el tema metropolis


%%%%%%%%%%%%%%%%%%%%%%%%%%%%%%%%%%%%%%%%%%%%%%%%%%%%%%%%%%%%%%%%%%%%%%%%%%%%%%%%
% Configurar footer
%\setbeamertemplate{frame footer}{\insertshortauthor \quad \quad \insertshortsubtitle}
%\setbeamercolor{footline}{fg=gray}

%%%%%%%%%%%%%%%%%%%%%%%%%%%%%%%%%%%%%%%%%%%%%%%%%%%%%%%%%%%%%%%%%%%%%%%%%%%%%%%%
% Portada
\title{Desarrollo de una arquitectura reactiva y deliberativa usando
planificación en el entorno de juegos GVGAI}
\subtitle{Trabajo Fin de Grado}
\date{\today}
\author{Vladislav Nikolov Vasilev}
\institute{Escuela Técnica Superior de Ingenierías Informática y de Telecomunicación\\
    Universidad de Granada}

\begin{document}
    % Insertar pagina de titulo
    \maketitle

    % Insertar indice
    \begin{frame}{Índice}
        \setbeamertemplate{section in toc}[sections numbered]
        \tableofcontents
    \end{frame}

    \section{Introducción}
    \begin{frame}{Motivación}
        Planificación automática como rama de la IA para la obtención de planes que permiten
        resolver problemas.

        Planificación integrada exitosamente en aplicaciones reales pero no en videojuegos.

        Desarrollo de agentes para un juego determinado cuya componente deliberativa se basa
        en planificación. 
    \end{frame}

    \begin{frame}{Objetivos}
        Crear una arquitectura en el entorno de juegos GVGAI con las siguientes características:

        \begin{enumerate}
            \item Combina componente reactiva con deliberativa basada en planificación.
            \item Lo suficientemente general para resolver cualquier juego del entorno.
        \end{enumerate}
    \end{frame}

    \begin{frame}{Contribuciones principales}
        \begin{itemize}
            \item Nuevas vías para experimentar con técnicas de planificación en GVGAI.
            \item Herramienta educativa.
        \end{itemize}
    \end{frame}

    \section{Antecedentes}

    \begin{frame}{Trabajos relacionados}

    \end{frame}

    \section{Plan de trabajo}
    \section{Arquitectura general del sistema}
    \section{Módulo de planificación}
    \section{Módulo de ejecución y monitorización}
    \section{Módulo de interacción con el usuario}
    \section{Implementación}
    \section{Experimentación}
    \section{Conclusiones}
\end{document}
