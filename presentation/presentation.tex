\documentclass[11pt]{beamer}    % Presentacion
\usepackage{graphicx}


\usetheme{metropolis}           % Usar el tema metropolis


%%%%%%%%%%%%%%%%%%%%%%%%%%%%%%%%%%%%%%%%%%%%%%%%%%%%%%%%%%%%%%%%%%%%%%%%%%%%%%%%
% Configurar footer
%\setbeamertemplate{frame footer}{\insertshortauthor \quad \quad \insertshortsubtitle}
%\setbeamercolor{footline}{fg=gray}

%%%%%%%%%%%%%%%%%%%%%%%%%%%%%%%%%%%%%%%%%%%%%%%%%%%%%%%%%%%%%%%%%%%%%%%%%%%%%%%%
% Portada
\title{Desarrollo de una arquitectura reactiva y deliberativa usando
planificación en el entorno de juegos GVGAI}
\subtitle{Trabajo Fin de Grado}
\date{\today}
\author{Vladislav Nikolov Vasilev}
\institute{Escuela Técnica Superior de Ingenierías Informática y de Telecomunicación\\
    Universidad de Granada}

\begin{document}
    % Insertar pagina de titulo
    \maketitle

    % Insertar indice
    \begin{frame}{Índice}
        \setbeamertemplate{section in toc}[sections numbered]
        \tableofcontents
    \end{frame}

    \section{Introducción}
    \begin{frame}{First Frame}
    Hello, world! adfasd adsfasff
    \end{frame}
    \section{Antecedentes}
    \section{Plan de trabajo}
    \section{Arquitectura general del sistema}
    \section{Módulo de planificación}
    \section{Módulo de ejecución y monitorización}
    \section{Módulo de interacción con el usuario}
    \section{Implementación}
    \section{Experimentación}
    \section{Conclusiones}
\end{document}
