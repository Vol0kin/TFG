%%%%%%%%%%%%%%%%%%%%%%%%%%%%%%%%%%%%%%%%%%%%%%%%%%%%%%%%%%%%%%%%%%%%%%%%%%%%%%%%
%                    Capitulo 3: Plan de trabajo                               %
%%%%%%%%%%%%%%%%%%%%%%%%%%%%%%%%%%%%%%%%%%%%%%%%%%%%%%%%%%%%%%%%%%%%%%%%%%%%%%%%

\chapter{Plan de trabajo}


\section{Metodología de trabajo seguida}

Desde el comienzo del proyecto se ha seguido una metodología de desarrollo ágil basada en
Scrum. Cada semana o cada dos semanas se celebraba una reunión, la cual se realizaba de manera
presencial o telemática mediante Skype o Google Meet. En dichas reuniones se exponía el trabajo
desarrollado desde la última reunión, los problemas que habían surgido durante su desarrollo, se
proponían algunas posibles soluciones para dichos problemas y se proponían una nueva serie de
objetivos que debían cumplirse antes de la siguiente reunión. En dichos nuevos objetivos se
consideraban, como era de esperar, los imprevistos que podían haber surgido durante el desarrollo
de los objetivos anteriores.

Seguir una metodología ágil como ésta ha facilitado el desarrollo del trabajo de manera sustancial.
Gracias a ella se descompuso un trabajo que a priori era bastante grande en pequeños objetivos semanales,
los cuales eran más abarcables y servían como guía para no perder de vista el objetivo principal del proyecto.

\section{Temporización}

En la figura \ref{fig:gantt-chart} se muestra la temporización seguida a lo largo del casi año que ha
durado este proyecto. En dicho diagrama aparecen reflejadas las tareas que se han llevado a cabo durante
este tiempo, las cuales se han agrupado según el tipo de trabajo que había que realizar.

\begin{figure}[H]
    \centering
    \scalebox{0.5}{
\begin{ganttchart}[%Specs
     y unit title=0.5cm,
     y unit chart=0.7cm,
     vgrid,hgrid,
     title height=1,
     title label font=\bfseries\footnotesize,
     bar/.style={fill=blue},
     bar height=0.4,
     group right shift=0,
     group top shift=0.7,
     group height=.3,
     group peaks width={0.2},
     inline]{1}{36}
    %labels
    \gantttitle{Jul}{3}
    \gantttitle{Ago}{3}
    \gantttitle{Sep}{3}
    \gantttitle{Oct}{3}
    \gantttitle{Nov}{3}
    \gantttitle{Dic}{3}
    \gantttitle{Ene}{3}
    \gantttitle{Feb}{3}
    \gantttitle{Mar}{3}
    \gantttitle{Abr}{3}
    \gantttitle{May}{3}
    \gantttitle{Jun}{3}\\
    % Setting group if any
    \ganttgroup[inline=false]{Estudio}{1}{35}\\
    \ganttbar[inline=false, bar/.style={fill=red}]{Planificación}{1}{4}\\
    \ganttbar[inline=false, bar/.style={fill=red}]{Arquitecturas de planificación}{15}{15}\\
    \ganttbar[inline=false, bar/.style={fill=red}]{Bibliografía complementaria}{34}{35}\\
    
    \ganttgroup[inline=false]{Diseño}{2}{30}\\
    \ganttbar[inline=false, bar/.style={fill=yellow}]{Definición dominio PDDL inicial}{2}{3}\\
    \ganttbar[inline=false, bar/.style={fill=yellow}]{Almacenamiento y gestión de objetivos}{15}{17}
    \ganttbar[inline=false, bar/.style={fill=yellow}]{}{21}{21}\\
    \ganttbar[inline=false, bar/.style={fill=yellow}]{Arquitectura del agente}{21}{22}\\
    \ganttbar[inline=false, bar/.style={fill=yellow}]{Archivo de configuración del sistema}{24}{26}\\
    \ganttbar[inline=false, bar/.style={fill=yellow}]{Definición dominios PDDL nuevos juegos}{27}{28}\\
    \ganttbar[inline=false, bar/.style={fill=yellow}]{Funcionalidad del modo depuración}{29}{30}\\
    
    \ganttgroup[inline=false]{Implementación}{1}{32}\\
    \ganttbar[inline=false, bar/.style={fill=cyan}]{Integración de planificador en GVGAI}{1}{3}\\
    \ganttbar[inline=false, bar/.style={fill=cyan}]{Traducción automática de estados del juego}{5}{12}\\
    \ganttbar[inline=false, bar/.style={fill=cyan}]{Gestión de objetivos}{16}{17}
    \ganttbar[inline=false, bar/.style={fill=cyan}]{}{21}{23}\\
    \ganttbar[inline=false, bar/.style={fill=cyan}]{Control de discrepancias}{21}{23}\\
    \ganttbar[inline=false, bar/.style={fill=cyan}]{Generación automática de archivos de configuración}{26}{27}\\
    \ganttbar[inline=false, bar/.style={fill=cyan}]{Modo depuración}{30}{32}\\
    
    \ganttgroup[inline=false]{Validación}{33}{33}\\
    \ganttbar[inline=false, bar/.style={fill=green}]{Creación de tests}{33}{33}\\
    
    \ganttgroup[inline=false]{Experimentación}{29}{35}\\
    \ganttbar[inline=false, bar/.style={fill=pink}]{Estudio del control de discrepancias}{29}{30}\\
    \ganttbar[inline=false, bar/.style={fill=pink}]{Estudio de la generación de problemas}{35}{35}\\
    
    \ganttgroup[inline=false]{Documentación}{22}{36} \\
    \ganttbar[inline=false, bar/.style={fill=orange}]{Código fuente}{22}{33}\\
    \ganttbar[inline=false, bar/.style={fill=orange}]{Memoria}{34}{36}
\end{ganttchart}}
    \caption{Diagrama de Gantt que muestra la temporización del proyecto.}
    \label{fig:gantt-chart}
\end{figure}

\subsection{Estudio}

El estudio fue una tarea que se llevó a cabo durante casi todo el ciclo de vida del proyecto.

Primero se hizo un estudio de algunos trabajos que trataban el tema de la planificación, además de
estudiar el funcionamiento de diversos planificadores para elegir el más adecuado para el problema
que se intentaba resolver, aunque al final, para crear el sistema, nos decantamos por otro planificador
que no se estudió en esta fase pero sí que era conocido.

Después, a medida que el proyecto iba tomando forma, se hizo un estudio de una arquitectura reactiva
y deliberativa basada en planificación que había sido propuesta por el director del proyecto. Dicha
arquitectura sirvió como base para desarrollar la arquitectura final que hemos propuesto en este trabajo.

Por último, a la vez que se comenzó con la redacción de este documento, se llevó a cabo un estudio
de otros trabajos relacionados con éste, de forma que se pudiera ver dónde se situaba la propuesta
de sistema que habíamos hecho con respecto a las propuestas que han hecho otros autores en el ámbito
de la planificación en videojuegos. También se realizó la consulta de diversas fuentes, las cuales
permitieron obtener información para la redacción de la parte de antecedentes presentada en el capítulo
anterior.

\subsection{Diseño}

El diseño comenzó con la creación del primer dominio de planificación \texttt{PDDL}, el cual
se usó en las primeras fases del proyecto. A medida que el proyecto iba tomando forma, y gracias
al diseño propuesto por el director, se comenzó con el diseño de una estructura de datos que permitiese
la gestión de los objetivos de manera sencilla. Solapándose con este proceso se comenzó también
el desarrollo de la arquitectura general del agente, ya que ya se tenía una base lo suficientemente
grande como para definirla.

Una vez hecho esto, se comenzó con el diseño del archivo de configuración, el cual juega un papel
muy importante en el sistema. Para ello se consideraron diversas propuestas, tanto de formato
como de generación, hasta que se dio con el formato y el método más adecuados.

Posteriormente, teniendo ya una forma de generar la información necesaria para ejecutar otros juegos
de manera sencilla, se comenzó el diseño de dominios de planificación \texttt{PDDL} para una serie
de nuevos juegos.

Finalmente, a la vista de que el sistema estaba en sus últimas fases, se comenzó con el diseño de
un modo depuración, el cuál permitiría obtener más información mientras se ejecutaba un juego.

\subsection{Implementación}

La implementación es, sin lugar a dudas, la tarea que más tiempo ha requerido a lo largo de
todo el desarrollo.

En una primera fase, el objetivo era integrar un planificador dentro de \texttt{GVGAI} para un
juego en concreto, determinando manualmente la traducción de los elementos del juego a predicados
\texttt{PDDL} dentro del código. De esta forma, se podían generar problemas de planificación \texttt{PDDL}
a partir de las observaciones del juego.

En una segunda fase se propuso conseguir una traducción automática de los estados del juego
a predicados \texttt{PDDL}, la cual permitiese al usuario definir cómo se traducía cada elemento
del juego, desacoplando por tanto dicha correspondencia entre elementos del juego y predicados \texttt{PDDL}
del código. Este objetivo suponía un primer paso hacia la creación de una arquitectura que permitiese
resolver múltiples juegos. En esta primera propuesta se utilizaron una serie de archivos en formato
\texttt{JSON} que definían un conjunto de correspondencias que permitían la traducción automática
de estados de observación del juego.

Una vez conseguido esto, se comenzó a trabajar en la gestión automática de objetivos, aunque todavía no
se habían conseguido desacoplar del código. Paralelamente a esto se comenzó a implementar el control de
discrepancias en el sistema. De esta forma, se consiguió crear una estructura de datos que pudiese
gestionar los objetivos en función de lo que sucedía dentro del juego.

Después de estudiar algunos diseños para el archivo de configuración, se comenzó con la implementación
de un generador que permitiese obtener un archivo de configuración plantilla a partir de un dominio
de planificación \texttt{PDDL} especificado por el usuario y una descripción en \texttt{VGDL} del
juego. Es aquí donde finalmente se desacoplaron los objetivos del código, permitiendo al usuario
establecer los objetivos que quiera.

En las últimas fases se implementó un modo depuración, el cual permitiría obtener maś información
del sistema mientras se estuviese ejecutando el juego.

\subsection{Validación}

Una vez que el sistema fue completamente implementado, se realizó una validación de éste creando
una serie de tests unitarios que debían pasar distintos componentes del sistema. Estos tests permitieron
comprobar el correcto funcionamiento del código implementado, además de que gracia a ellos se pudieron
detectar algunos fallos menores que habían pasado desapercibidos.

\subsection{Experimentación}

Teniendo el sistema casi totalmente implementado, se hizo un primer estudio para ver qué tal respondía
el agente a las discrepancias que se podían dar a la hora de ejecutar un plan. Posteriormente, se llevó
a cabo un segundo estudio que pretendía poner de manifiesto la rapidez del sistema para generar problemas
de planificación \texttt{PDDL} a partir de estados de observación del juego.

\subsection{Documentación}

Por último, resta hablar de la documentación. Este es un proceso muy importante en cualquier
proyecto, ya que permite conocer qué se ha llevado a cabo y cómo se ha desarrollado el proyecto.

La tarea de documentación comenzó con la documentación del código fuente. Se decidió comenzar
con dicha tarea cuando una gran parte del sistema ya estaba implementado, de forma que no fuese
necesario reescribirla demasiado.

Esta tarea concluyó con la redacción de este documento, el cual pone de manifiesto todo el trabajo
que ha sido desarrollado desde el comienzo del proyecto hasta el final del mismo.
