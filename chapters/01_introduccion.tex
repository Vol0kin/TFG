%%%%%%%%%%%%%%%%%%%%%%%%%%%%%%%%%%%%%%%%%%%%%%%%%%%%%%%%%%%%%%%%%%%%%%%%%%%%%%%%
%                    Capitulo 1: Introduccion                                  %
%%%%%%%%%%%%%%%%%%%%%%%%%%%%%%%%%%%%%%%%%%%%%%%%%%%%%%%%%%%%%%%%%%%%%%%%%%%%%%%%

\chapter{Introducción}

La planificación automática es una rama de la Inteligencia Artificial cuyo objetivo
último es proporcionar planes que permitan resolver problemas \cite{10.5555/3073924}.
Para ello se hace uso de un \textbf{\textit{planificador}}, que es un programa que recibe
como entradas un \textbf{\textit{dominio de planificación}}, es decir, una representación
del mundo, y un \textbf{\textit{problema de planificación}}, que es un estado concreto de
ese mundo que debe ser resuelto. El resultado obtenido es una secuencia ordenada de acciones
que permiten alcanzar el objetivo descrito en el problema desde el estado
inicial que se describe en él.

A pesar de que la planificación automática ha sido integrada con éxito en muchas aplicaciones
reales como por ejemplo la robótica o los vehículos no tripulados, el campo de los videojuegos
sigue suponiéndole un gran reto. Esto se debe a que los videojuegos presentan, en líneas generales,
entornos dinámicos y complejos, los cuales requieren una respuesta rápida ante los cambios que se producen
y presentan espacios de búsqueda muy grandes debido al elevado número de acciones que existen en éstos.
El proceso de planificación es costoso y lento, por lo que los planificadores no se ven
capaces de ofrecer una respuesta rápida a los cambios que se producen en su entorno.

El reto que supone integrar una arquitectura deliberativa basada en planificación en un videojuego ha
abierto muchas líneas de investigación. La mayoría de ellas se centran en torno al juego
StarCraft \cite{10.1007/978-3-540-74141-1_12, Churchill2011BuildOO, Weber2011BuildingHA, Aha_2018}
debido a la enorme dinamicidad y complejidad que se puede encontrar en dicho juego.
Existen, no obstante, otros juegos que se podrían utilizar para dichas pruebas.
\texttt{GVGAI} (\textit{General Video Game AI})\cite{7038214} es un entorno que
dispone de más de cien juegos diferentes. Entre ellos existen juegos que pueden ser resueltos
muy fácilmente mediante técnicas de planificación, y también hay otros juegos con entornos
dinámicos que pueden llegar a suponer un auténtico reto para los planificadores.

Si bien es cierto que \texttt{GVGAI} tiene una gran cantidad de juegos que pueden resultar
interesantes desde el punto de vista de la planificación, hay que tener en cuenta que el entorno
no fue concebido para la creación de arquitecturas deliberativas basadas en técnicas de planificación.
Principalmente, el entorno surgió para permitir el desarrollo y la competición de arquitecturas reactivas basadas en
técnicas como el Aprendizaje por Refuerzo o MCTS (\textit{Monte Carlo Tree Search}). Por tanto, crear agentes
basados en planificación para los distintos juegos de \texttt{GVGAI} puede llegar a ser un proceso costoso que además
se sale de la propuesta original del entorno. Es precisamente en este contexto que surge el presente trabajo.

El objetivo principal de este trabajo es la creación de una arquitectura que combina un componente
reactivo con uno deliberativo basado en planificación en el entorno de juegos \texttt{GVGAI}. Además,
a diferencia de las anteriores arquitecturas, las cuales estaban creadas para un juego en específico,
la propuesta que se hace en este trabajo tiene también como objetivo ser lo suficientemente general como
para permitir resolver cualquier juego del entorno que se desee mediante planificación, siempre y
cuando éste pueda ser resuelto utilizando este tipo de técnicas.

Esta propuesta puede llegar a ser bastante útil por dos motivos principales. Por un
lado abre nuevas vías para experimentar con técnicas de planificación en \texttt{GVGAI}, lo cual es de
sumo interés para el desarrollo de arquitecturas \textit{online} que integran un planificador que
dirige el comportamiento deliberativo de un agente. Por otro lado, puede utilizarse como herramienta
docente para enseñar cómo se representan los problemas de planificación y para comprender mejor el funcionamiento
de la planificación.

El trabajo se ha estructurado de la siguiente manera. En el capítulo 2 se hará
una introducción, comentando los trabajos relacionados e introduciendo conceptos
fundamentales como la planificación y \texttt{GVGAI}. Después, en el capítulo 3 se
detallará el plan de trabajo seguido para el desarrollo del proyecto. Una vez hecho esto,
en el capítulo 4 se presentará la arquitectura general del sistema, y seguidamente, en el
capítulo 5 se comentará la arquitectura en más detalle. En el capítulo 6 se hablará de la creación
y del formato del archivo de configuración. Una vez hecha la explicación pasaremos al capítulo 7,
donde se hará una descripción general de la implementación \textit{software} que se
ha llevado a cabo. En el capítulo 8 se presentarán una serie de experimentos que
se han realizado con el sistema. Por último, en el capítulo 9 se presentarán las
conclusiones que se pueden extraer y trabajos futuros.