%%%%%%%%%%%%%%%%%%%%%%%%%%%%%%%%%%%%%%%%%%%%%%%%%%%%%%%%%%%%%%%%%%%%%%%%%%%%%%%%
%                    Capitulo 1: Introduccion                                  %
%%%%%%%%%%%%%%%%%%%%%%%%%%%%%%%%%%%%%%%%%%%%%%%%%%%%%%%%%%%%%%%%%%%%%%%%%%%%%%%%

\chapter{Introducción}

El trabajo se ha estructurado de la siguiente manera. En el capítulo 2 se hará
una introducción, comentando los trabajos relacionados e introduciendo conceptos
fundamentales como la planificación y \texttt{GVGAI}. Después, en el capítulo 3 se
detallará el plan de trabajo seguido, esto es, la metodología de trabajo y la
temporización. Una vez hecho esto, en el capítulo 4 se presentará la arquitectura
general del sistema, y seguidamente, en el capítulo 5 se comentará la arquitectura
en más detalle. En el capítulo 6 se hablará del archivo de configuración, de cómo
crearlo y qué formato tiene. Una vez hecha la explicación pasaremos al capítulo 7,
donde se hará una descripción general de la implementación \textit{software} que se
ha llevado a cabo. En el capítulo 8 se presentarán una serie de experimentos que
se han realizado con el sistema. Por último, en el capítulo 9 se presentarán las
conclusiones a las que se han llegado y trabajos futuros.