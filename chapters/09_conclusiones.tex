%%%%%%%%%%%%%%%%%%%%%%%%%%%%%%%%%%%%%%%%%%%%%%%%%%%%%%%%%%%%%%%%%%%%%%%%%%%%%%%%
%                       Capitulo 9: Conclusiones                               %
%%%%%%%%%%%%%%%%%%%%%%%%%%%%%%%%%%%%%%%%%%%%%%%%%%%%%%%%%%%%%%%%%%%%%%%%%%%%%%%%

\chapter{Conclusiones}

En este trabajo hemos propuesto una arquitectura semiautomática que combina un componente
deliberativo guiado por la planificación y un componente reactivo que permite la ejecución y
monitorización de los planes obtenidos, detectando en el proceso cualquier tipo de situación
imprevista que pueda surgir.

Tal y como se ha podido ver, esta nueva arquitectura, a diferencia de otras propuestas diseñadas para
manejar un único dominio de planificación para un único juego, permite resolver distintos juegos mediante
el uso de planificación. Esto es posible siempre y cuando dichos juegos puedan ser resueltos mediante este
tipo de técnicas. Esto se ha conseguido gracias a la definición de un proceso semiautomático para la
integración del planificador, basado en archivos de configuración que permiten al usuario establecer tanto
los parámetros que va a utilizar el sistema para traducir estados de observación del juego a predicados
\texttt{PDDL} como los objetivos que se desea alcanzar en la ejecución del sistema.

Un punto fuerte de esta arquitectura es que simplifica mucho la creación de problemas
de planificación \texttt{PDDL}, ya que permite obtenerlos automáticamente a partir
de los estados de observación del juego y en muy poco tiempo, como se ha podido comprobar, incluso
para problemas grandes. Si se intentase hacer el proceso de integración de un planificador en un juego,
el cual requiere de la definición de un dominio y de la generación de problemas específicos para el juego,
se podrían tardar días, e incluso semanas.

Por último, cabe destacar que esta propuesta puede ser utilizada en dos ámbitos. Por una parte, puede
utilizarse para experimentar con arquitecturas deliberativas dirigidas por un planificador en el entorno
\texttt{GVGAI}. Por otra, puede ser utilizada con fines educativos, permitiendo a los estudiantes entender la
generación de problemas a partir de dominios de planificación \texttt{PDDL} y comprender mejor cómo funciona
la planificación.

\section{Trabajos futuros}

A pesar de los éxitos logrados en este trabajo, se considera que todavía hay cierto margen
de mejora. En este punto se proponen una serie de ideas que permitirían mejorar el sistema y
que pueden servir como trabajos futuros. Es importante destacar que estas propuestas tienen
que permitir la resolución de múltiples juegos, de forma que integrar estos cambios en el
sistema puede suponer un auténtico reto. Las propuestas de mejora que se hacen son las siguientes:

\begin{itemize}[label=\textbullet]
    \item \textbf{Mejora del comportamiento reactivo}. El comportamiento reactivo que se
    presenta en este trabajo es simple. Cuando el agente va a ejecutar la siguiente acción,
    comprueba que sus precondiciones se cumplen, y en caso de que no, cambia de objetivo.
    Este comportamiento, a pesar de que funciona en la mayoría de los casos, tiene el problema
    de que no tiene en mayor consideración el entorno del agente. Muchas veces, el agente
    puede llegar a morir aun sin haberse producido una discrepancia, debido a que por ejemplo
    ha aparecido un enemigo que no ha podido detectar al comprobar las precondiciones de la acción.
    Para solventar esto, se puede modificar el comportamiento reactivo del agente, incluyendo alguno
    de los controladores reactivos que se proporcionan en \texttt{GVGAI} o creando uno propio. De esta
    forma, el comportamiento reactivo podría permitir sortear peligros que se detecten en el entorno
    del agente, y después aplicar algún método para reparar el plan obtenido por el planificador.
    \item \textbf{Integración de un módulo de \textit{goal reasoning}}. Muchas nuevas propuestas de
    arquitecturas deliberativas basadas en planificación incluyen un módulo de \textit{goal reasoning},
    como por ejemplo la que se puede ver en \cite{Aha_2018}. Este módulo permite generar objetivos
    de forma automática a partir del estado del juego. Dichos objetivos son comunicados al planificador,
    el cual se encarga de encontrar un plan hasta ellos. Puede resultar de interés integrar un módulo
    de este tipo en el sistema ya que, si la arquitectura es capaz de proponerse ella misma los objetivos
    de forma automática, el usuario solo tendría que especificar qué predicados del dominio de planificación
    \texttt{PDDL} que ha definido se pueden considerar como objetivo y el módulo, a partir de esta
    información y de la información de traducción proporcionada, podría plantear objetivos de forma automática.
    En esta descripción, sin embargo, se omite todo el trabajo que supone entrenar un módulo de este
    tipo, además de hacerlo lo más general posible para que pueda resolver múltiples juegos.
\end{itemize}