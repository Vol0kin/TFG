%%%%%%%%%%%%%%%%%%%%%%%%%%%%%%%%%%%%%%%%%%%%%%%%%%%%%%%%%%%%%%%%%%%%%%%%%%%%%%%%
%               Capitulo 5: Descripción detallada del sistema                  %
%%%%%%%%%%%%%%%%%%%%%%%%%%%%%%%%%%%%%%%%%%%%%%%%%%%%%%%%%%%%%%%%%%%%%%%%%%%%%%%%

\chapter{Descripción detallada del sistema}

Una vez que se ha visto cuál es la arquitectura general del sistema y cómo se agrupan
y relacionan los diferentes módulos lógicos y componentes funcionales, vamos a pasar
a estudiar dichos elementos de manera más detallada, haciendo especial hincapié en los
componentes funcionales.

\section{Módulo de interacción con el usuario}

El módulo de interacción con el usuario, tal y como se comentó en el capítulo anterior,
en este módulo se lleva a cabo la interacción con el usuario

\subsection{Creación del dominio}

\subsection{Creación de la configuración}

\section{Módulo de planificación}

\subsection{Gestor de objetivos}

\subsection{Generador de problemas}

\subsection{Planificador}

\subsection{Traductor de planes}

\section{Módulo de ejecución del plan y monitorización}

\subsection{Motor del juego}

\subsection{Traductor del estado de observación}

\subsection{Monitor}

\subsection{Ejecutor del plan}

