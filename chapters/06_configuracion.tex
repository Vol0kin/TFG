%%%%%%%%%%%%%%%%%%%%%%%%%%%%%%%%%%%%%%%%%%%%%%%%%%%%%%%%%%%%%%%%%%%%%%%%%%%%%%%%
%           Capitulo 6: Descripcion del archivo de configuracion               %
%%%%%%%%%%%%%%%%%%%%%%%%%%%%%%%%%%%%%%%%%%%%%%%%%%%%%%%%%%%%%%%%%%%%%%%%%%%%%%%%

\chapter{Descripción del archivo de configuración}

Como se ha podido ver hasta ahora, el archivo de configuración definido por el usuario
juega un papel clave en el sistema. Es gracias a él que la mayoría de operaciones dentro
del sistema pueden ser automatizadas, como por ejemplo la generación de problemas, la
traducción de estados de observación a estados \texttt{PDDL}, la gestión de objetivos
y la traducción de planes. Por tanto, es importante saber cómo se crea dicho fichero
y qué formato tiene.

\section{Creación del archivo}

Como se comentó brevemente en el capítulo anterior, la creación del archivo de configuración
está semiautomatizada. Se tiene un \textit{script} que, dado un archivo que contiene el dominio
\texttt{PDDL} y el archivo de descripción del juego en formato \texttt{VGDL}, genera un archivo
de configuración plantilla para que el usuario lo pueda rellenar. Algunos de los campos de
ese archivo ya estarán rellenados, pero se deja total libertad al usuario para que los modifique.
También existen ciertos campos que son opcionales, aunque se concretará más cuáles son cuando
se describa la estructura del archivo.

\section{Formato del archivo de configuración}

El archivo de configuración está en formato \texttt{YAML} (\textit{YAML Ain't Markup Language}),
que es un lenguaje para la serialización de datos creado de forma que sea muy fácil de leer tanto
por humanos como por la máquina. Es esta legibilidad lo que hace que \texttt{YAML} sea ideal para
la creación de archivos de configuración, ya que de manera sencilla y clara se pueden establecer
los parámetros que configuran un sistema.

El lenguaje está formado por tipos básicos como cadenas de texto, enteros, números reales y valores
booleanos. También tiene tipos más avanzados, como listas y relaciones clave-valor.

\section{Estructura del archivo de configuración}

Para explicar la estructura del archivo vamos a partir de un archivo de configuración plantilla
generado para el juego \textit{Boulderdash}. El juego como tal se explicará en capítulos posteriores.
A partir de este archivo plantilla se explicarán los campos que lo forman y cómo se rellenarían dichos
campos.

\begin{lstlisting}[language=yaml]
domainFile: domains/boulderdash-domain.pddl
problemFile: problem.pddl
domainName: BoulderDash
cellVariable: null
avatarVariable: null
gameElementsCorrespondence:
  background:
  - null
  wall:
  - null
  sword:
  - null
  dirt:
  - null
  exitdoor:
  - null
  diamond:
  - null
  boulder:
  - null
  avatar:
  - null
  crab:
  - null
  butterfly:
  - null
variablesTypes:
  ?variable: Type
orientationCorrespondence:
  UP: null
  DOWN: null
  LEFT: null
  RIGHT: null
connections:
  UP: null
  DOWN: null
  LEFT: null
  RIGHT: null
actionsCorrespondence:
  TURN-UP: ACTION_UP
  TURN-DOWN: ACTION_DOWN
  TURN-LEFT: ACTION_LEFT
  TURN-RIGHT: ACTION_RIGHT
  MOVE-UP: ACTION_UP
  MOVE-DOWN: ACTION_DOWN
  MOVE-LEFT: ACTION_LEFT
  MOVE-RIGHT: ACTION_RIGHT
  MOVE-UP-GET-GEM: ACTION_UP
  MOVE-DOWN-GET-GEM: ACTION_DOWN
  MOVE-LEFT-GET-GEM: ACTION_LEFT
  MOVE-RIGHT-GET-GEM: ACTION_RIGHT
  DIG-UP: ACTION_UP
  DIG-DOWN: ACTION_DOWN
  DIG-LEFT: ACTION_LEFT
  DIG-RIGHT: ACTION_RIGHT
  EXIT-LEVEL: null
goals:
- goalPredicate: null
  priority: 0
  saveGoal: false
  removeReachedGoalsList:
  - null
\end{lstlisting}

Los campos que se generan de forma automática son \texttt{domainFile}, \texttt{problemFile},
\texttt{domainName} y \texttt{actionsCorrespondence}.