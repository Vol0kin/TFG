%%%%%%%%%%%%%%%%%%%%%%%%%%%%%%%%%%%%%%%%%%%%%%%%%%%%%%%%%%%%%%%%%%%%%%%%%%%%%%%%
%           Capitulo 6: Descripcion del archivo de configuracion               %
%%%%%%%%%%%%%%%%%%%%%%%%%%%%%%%%%%%%%%%%%%%%%%%%%%%%%%%%%%%%%%%%%%%%%%%%%%%%%%%%

\chapter{Descripción del archivo de configuración}

Como se ha podido ver hasta ahora, el archivo de configuración definido por el usuario
juega un papel clave en el sistema. Es gracias a él que la mayoría de operaciones dentro
del sistema pueden ser automatizadas, como por ejemplo la generación de problemas, la
traducción de estados de observación a estados \texttt{PDDL}, la gestión de objetivos
y la traducción de planes. Por tanto, es importante saber cómo se crea dicho fichero
y qué formato tiene.

\section{Creación del archivo}

Como se comentó brevemente en el capítulo anterior, la creación del archivo de configuración
está semiautomatizada. Se tiene un \textit{script} que, dado un archivo que contiene el dominio
\texttt{PDDL} y el archivo de descripción del juego en formato \texttt{VGDL}, genera un archivo
de configuración plantilla para que el usuario lo pueda rellenar. Algunos de los campos de
ese archivo ya estarán rellenados, pero se deja total libertad al usuario para que los modifique.
También existen ciertos campos que son opcionales, aunque se concretará más cuáles son cuando
se describa la estructura del archivo.

\section{Formato del archivo de configuración}

El archivo de configuración está en formato \texttt{YAML} (\textit{YAML Ain't Markup Language}),
que es un lenguaje para la serialización de datos creado de forma que sea muy fácil de leer tanto
por humanos como por la máquina. Es esta legibilidad lo que hace que \texttt{YAML} sea ideal para
la creación de archivos de configuración, ya que de manera sencilla y clara se pueden establecer
los parámetros que configuran un sistema.

El lenguaje está formado por tipos básicos como cadenas de texto, enteros, números reales y valores
booleanos. También tiene tipos más avanzados, como listas y relaciones clave-valor.

\section{Estructura del archivo de configuración}

Para explicar la estructura del archivo vamos a partir de un archivo de configuración plantilla
generado para el juego \textit{Boulderdash}. El juego como tal se explicará en capítulos posteriores.
A partir de este archivo plantilla se explicarán los campos que lo forman y cómo se rellenarían dichos
campos.

A continuación se ofrece el contenido de la plantilla que se ha obtenido a la hora de ejecutar
el \textit{script}:

\begin{lstlisting}[language=yaml]
domainFile: domains/boulderdash-domain.pddl
problemFile: problem.pddl
domainName: BoulderDash
cellVariable: null
avatarVariable: null
gameElementsCorrespondence:
  background:
  - null
  wall:
  - null
  sword:
  - null
  dirt:
  - null
  exitdoor:
  - null
  diamond:
  - null
  boulder:
  - null
  avatar:
  - null
  crab:
  - null
  butterfly:
  - null
variablesTypes:
  ?variable: Type
orientationCorrespondence:
  UP: null
  DOWN: null
  LEFT: null
  RIGHT: null
connections:
  UP: null
  DOWN: null
  LEFT: null
  RIGHT: null
actionsCorrespondence:
  TURN-UP: ACTION_UP
  TURN-DOWN: ACTION_DOWN
  TURN-LEFT: ACTION_LEFT
  TURN-RIGHT: ACTION_RIGHT
  MOVE-UP: ACTION_UP
  MOVE-DOWN: ACTION_DOWN
  MOVE-LEFT: ACTION_LEFT
  MOVE-RIGHT: ACTION_RIGHT
  MOVE-UP-GET-GEM: ACTION_UP
  MOVE-DOWN-GET-GEM: ACTION_DOWN
  MOVE-LEFT-GET-GEM: ACTION_LEFT
  MOVE-RIGHT-GET-GEM: ACTION_RIGHT
  DIG-UP: ACTION_UP
  DIG-DOWN: ACTION_DOWN
  DIG-LEFT: ACTION_LEFT
  DIG-RIGHT: ACTION_RIGHT
  EXIT-LEVEL: null
goals:
- goalPredicate: null
  priority: 0
  saveGoal: false
  removeReachedGoalsList:
  - null
\end{lstlisting}

Los campos que aparecen en el fichero son los siguientes:

\begin{enumerate}
    \item \texttt{domainFile}: Nombre del fichero de dominio \texttt{PDDL}. Este campo se genera
    automáticamente a partir de los parámetros utilizados a la hora de ejecutar el \textit{script}.
    
    \item \texttt{problemFile}: Nombre del fichero de problema de salida. Tiene como valor por defecto
    \texttt{problem.pddl}, pero el usuario puede darle el nombre que quiera.
    
    \item \texttt{domainName}: Nombre del dominio. Se genera automáticamente y se obtiene a partir del
    archivo de dominio \texttt{PDDL}.
    
    \item \texttt{cellVariable}: Variable que se utilizará para representar las celdas. Una variable
    viene precedida por el símbolo ``?'' y se le da el nombre que quiera el usuario. El uso del símbolo
    de interrogación es para que tenga cierta semejanza con la manera en la que se hacen las definiciones
    de variables en \texttt{PDDL}. Por ejemplo, en este caso se que la variable que representa una celda
    sea la siguiente:
    
    \begin{lstlisting}[language=yaml]
cellVariable: ?c
    \end{lstlisting}
    
    Por tanto, cuando el traductor se encuentre que para un elemento del juego alguno de los predicados
    asociados contiene la variable \texttt{?c} lo sustituirá por \texttt{c\_x\_y}, donde \texttt{x} es
    la columna e \texttt{y} la fila donde se encuentra dicho elemento del juego.
    
    \item \texttt{avatarVariable}: Variable que se utilizará para representar al avatar/agente. Por
    ejemplo, en este caso se ha decidido que sea la siguiente:
    
    \begin{lstlisting}[language=yaml]
avatarVariable: ?p
    \end{lstlisting}
    
    Esta variable es especial, ya que cuando se encuentre dicha variable en alguno de los predicados asociados
    al avatar se va a sustituir simplemente por el nombre que le ha dado el usuario. Por ejemplo,
    en este caso se sustituiría por \texttt{p}. Se ha hecho de esta forma debido a que simplifica
    mucho la monitorización del plan, ya que tener una variable que constantemente cambia sus coordenadas
    $x,y$ solo induciría a errores.
    
    \item \texttt{gameElementsCorrespondence}: Este campo sirve para asociar una lista de predicados
    a cada observación (elemento) del juego, de forma que cuando el traductor de estados de observación
    se encuentre con dicha observación, generará los correspondientes predicados instanciando las variables.
    Las observaciones se obtienen a partir del fichero que define el juego en formato \texttt{VGDL}, el
    cual recordemos que se pasa como parámetro a la hora de generar la plantilla del archivo de configuración.
    Una observación que aparece siempre es \texttt{background}, la cual representa una celda en la que no
    hay nada. Para este juego en concreto se han definido las siguientes correspondencias:
    
    \begin{lstlisting}[language=yaml]
gameElementsCorrespondence:
  background:
  - (terrain-empty ?c)
  wall:
  - (terrain-wall ?c)
  sword:
  - (terrain-empty ?c)
  dirt:
  - (terrain-ground ?c)
  exitdoor:
  - (at ?e ?c)
  - (terrain-empty ?c)
  diamond:
  - (at ?g ?c)
  - (terrain-empty ?c)
  - (occupied ?c)
  boulder:
  - (at ?boulder ?c)
  - (terrain-empty ?c)
  - (occupied ?c)
  avatar:
  - (at ?p ?c)
  - (terrain-empty ?c)
  crab:
  - (at ?s ?c)
  - (terrain-empty ?c)
  - (occupied ?c)
  butterfly:
  - (at ?b ?c)
  - (terrain-empty ?c)
  - (occupied ?c)    
    \end{lstlisting}
    
    Los predicados que aparecen, como es de suponer, deben ser los que se encuentran en el dominio
    \texttt{PDDL} y deben ser correctos. Otra cosa importante es que se debe mantener un mínimo de consistencia.
    Por ejemplo, si se ha definido que las celdas se representan mediante la variable \texttt{?c}, en
    todos aquellos predicados donde aparezcan celdas debería aparecer dicha variable. En resumen, la nomenclatura
    que se utiliza para definir las variables debe ser consistente, ya que si no se hace de esta forma,
    se pueden llegar a producir errores a la hora de traducir las observaciones del juego.
    
    \item \texttt{variablesTypes}: Este campo representa una correspondencia entre las variables que
    aparecen en los predicados del campo anterior y los tipos de \texttt{PDDL} que el usuario haya definido
    en el dominio. Para este juego en concreto se han definido las siguientes asociaciones:
    
    \begin{lstlisting}[language=yaml]
variablesTypes:
  ?e: Exit
  ?p: Player
  ?boulder: Boulder
  ?g: Gem
  ?b: Bat
  ?s: Scorpion
  ?c: Cell
    \end{lstlisting}
    
    Aunque anteriormente se ha definido cuáles son las variables asociadas a las celdas y al agente,
    sigue siendo necesario especificar su tipo, ya que el usuario podría darle cualquier nombre al
    tipo asociado a dichas variables al no haber un estándar.
    
    \item \texttt{orientationCorrespondence}: Este campo es opcional y solo se utiliza en aquellos
    juegos donde la orientación del personaje sea un factor a tener en cuenta. A cada clave se le
    tiene que asociar el correspondiente predicado que indique que el agente está orientado en esa
    dirección (\texttt{UP} para arriba, \texttt{DOWN} para abajo, \texttt{LEFT} para izquierda
    y \texttt{RIGHT} para derecha). En este caso, los siguientes predicados indican la orientación:
    
    \begin{lstlisting}[language=yaml]
orientationCorrespondence:
  UP: (oriented-up ?p)
  DOWN: (oriented-down ?p)
  LEFT: (oriented-left ?p)
  RIGHT: (oriented-right ?p) 
    \end{lstlisting}
    
    \item \texttt{connections}: Indica la relación de conectividad entre las celdas del juego.
    Esto obliga a que en el dominio se represente el mapa del juego como una cuadrícula. Por tanto, tienen
    que haberse declarado una serie de predicados que indiquen las conexiones entre las celdas. La clave
    \texttt{UP} indica que una celda está encima de la otra, \texttt{DOWN} que una está debajo de la otra,
    \texttt{LEFT} que una está a la izquierda de la otra y \texttt{RIGHT} que una está a la derecha de
    la otra. Un ejemplo de esto se puede ver a continuación:
    
    \begin{lstlisting}[language=yaml]
connections:
  UP: (connected-up ?c ?u)
  DOWN: (connected-down ?c ?d)
  LEFT: (connected-left ?c ?l)
  RIGHT: (connected-right ?c ?r)
    \end{lstlisting}
    
    Es importante destacar que las variables que aparecen en estos predicados no tienen nada que ver
    con las que se han definido en anteriores campos. Cuando se recorra el mapa para generar los relaciones
    de conectividad la variable \texttt{?c} representará la celda actual, la cual está en la posición $(x,y)$.
    La variable \texttt{?u} representará la celda superior a la actual situada en la posición $(x, y-1)$,
    \texttt{?d} la inferior respecto a la actual situada en la posición $(x, y+1)$, \texttt{?l} la izquierda
    respecto a la actual situada en la posición $(x-1, y)$ y \texttt{?r} la derecha respecto a la actual 
    situada en la posición $(x+1, y)$.
    
    \item \texttt{actionsCorrespondence}: Correspondencia entre las acciones definidas en el dominio
    \texttt{PDDL} y las acciones de \texttt{GVG-AI}. Las acciones \texttt{PDDL} se obtienen
    a partir del archivo de dominio. Si el nombre de la acción \texttt{PDDL} contiene ciertos
    patrones, se produce una traducción automática a una acción de \texttt{GVG-AI}:
    
    \begin{itemize}[label=\textbullet]
        \item Si contiene \textit{up}, la acción se traduciría como \texttt{ACTION\_UP}.
        \item Si contiene \textit{down}, la acción se traduciría como \texttt{ACTION\_DOWN}.
        \item Si contiene \textit{left}, la acción se traduciría como \texttt{ACTION\_LEFT}.
        \item Si contiene \textit{right}, la acción se traduciría como \texttt{ACTION\_RIGHT}.
        \item Si contiene \textit{use}, la acción se traduciría como \texttt{ACTION\_USE}.
        \item Si no contiene ninguno de los patrones anteriores, se traduciría como \texttt{null},
        dejándole al usuario la tarea de asignarle un acción de \texttt{GVG-AI}, si es que la hay.
    \end{itemize}
    
    Obviamente, el usuario puede modificar las traducciones a su gusto. Las acciones que tienen
    como traducción \texttt{null} no se tendrán en cuenta a la hora de traducir los planes. También,
    si no se quiere que se tenga en cuenta una acción, bien porque no tenga traducción o bien por otro
    motivo, simplemente bastaría con eliminar dicha acción.
    
    \item \texttt{goals}: Representa la lista de objetivos definida por el usuario. Un objetivo está compuesto
    por un predicado \texttt{PDDL} que representa lo que se quiere alcanzar, la prioridad del objetivo, una
    opción que indica si guardar el objetivo o no una vez que este ha sido alcanzado y un campo opcional
    que contiene una lista de objetivos alcanzados y guardados que deben ser eliminados. Cuanto menor
    sea el valor de la prioridad, más prioridad tendrá dicho objetivo. Puede darse el caso de que haya
    más de un objetivo con la misma prioridad, lo cual implica que no importa el orden en el que se
    completen dichos objetivos. Guardar un objetivo sirve para incluirlo en los próximos estados \texttt{PDDL}
    que genere el traductor de estados de observación, ya que puede ser que al alcanzar dicho objetivo
    se produzca un cambio en el juego que debe ser considerado para alcanzar objetivos posteriores.
    Eliminar un objetivo guardado una vez que se ha alcanzado otro implica que algo que anteriormente
    era verdad ha dejado de serlo, y por tanto, el objetivo a eliminar no será incluido en posteriores
    estados \texttt{PDDL}. Obviamente, para que esto suceda se tiene que haber alcanzado el objetivo a
    eliminar previamente. A continuación se puede ver un ejemplo de esto:
    
    \begin{lstlisting}[language=yaml]
goals:
- goalPredicate: (got g_5_3)
  priority: 1
  saveGoal: no
- goalPredicate: (got g_6_3)
  priority: 1
  saveGoal: no
- goalPredicate: (got g_7_3)
  priority: 1
  saveGoal: no
- goalPredicate: (got g_1_4)
  priority: 1
  saveGoal: no
- goalPredicate: (got g_6_1)
  priority: 1
  saveGoal: no
- goalPredicate: (got g_7_1)
  priority: 1
  saveGoal: no
- goalPredicate: (got g_7_9)
  priority: 1
  saveGoal: no
- goalPredicate: (got g_9_10)
  priority: 1
  saveGoal: no
- goalPredicate: (exited-level)
  priority: 2
  saveGoal: no
- goalPredicate: (got g_16_9)
  priority: 1
  saveGoal: no
    \end{lstlisting}
    
    Aquí se indica que el agente primero debe recoger 9 gemas en cualquier orden (debido a que tienen la misma
    prioridad) y que, una vez que las ha obtenido, debe salir del nivel. Ninguno de estos objetivos tiene que ser
    guardado, y ninguno tiene una lista de objetivos a eliminar una vez que se haya alcanzado. Es importante destacar
    que, a la hora de indicar los objetivos, las variables de los predicados se sustituyen por objetos (instancias
    concretas de las variables). Por ejemplo, si nos fijamos en el primer objetivo, vemos que el predicado con la
    variable instanciada es \texttt{(got g\_5\_3)} (hay una gema en la posición $x = 5, y= 7$ ), mientras que el
    predicado como tal sería \texttt{(got ?g)}. Por tanto, es necesario conocer cómo se instancian las variables,
    así como también la posición de los elementos en el mapa.
\end{enumerate}