%%%%%%%%%%%%%%%%%%%%%%%%%%%%%%%%%%%%%%%%%%%%%%%%%%%%%%%%%%%%%%%%%%%%%%%%%%%%%%%%
%                    Capitulo 2: Antecedentes                                  %
%%%%%%%%%%%%%%%%%%%%%%%%%%%%%%%%%%%%%%%%%%%%%%%%%%%%%%%%%%%%%%%%%%%%%%%%%%%%%%%%

\chapter{Antecedentes}

\section{Trabajos relacionados}

Hasta donde sabemos, no se ha desarrollado una arquitectura de planificación para
controlar el comportamiento deliberativo de un agente en \texttt{GVGAI}. No
obstante, se han desarrollado otras arquitecturas de planificación para otros juegos,
como es el caso de StarCraft. Algunas de estas arquitecturas son, por ejemplo,
\texttt{Darmok} \cite{10.1007/978-3-540-74141-1_12}, basada en casos;
\texttt{UalbertBot} \cite{Churchill2011BuildOO}, basada en la búsqueda heurística de
planes concurrentes; \texttt{EISBot} \cite{Weber2011BuildingHA}, basada en planificación reactiva;
y una arquitectura basada en \texttt{PELEA} \cite{Alczar2010peleaP} que combina
\textit{goal reasoning} \cite{Aha_2018} con planificación clásica.

A diferencia de las anteriores, nuestra arquitectura está diseñada para resolver
problemas en varios dominios de planificación y juegos en vez de solo en un
dominio de planificación y un juego, que en este caso sería StarCraft.
Por esa razón, presentamos una metodología semiautomática para la integración de un
planificador en cualquier juego de \texttt{GVGAI} que pueda ser resuelto mediante
planificación.

\section{Planificación automática}

La planificación automática


\section{\texttt{GVGAI}}

\texttt{GVGAI}

\subsection{\texttt{VGDL}}