%%%%%%%%%%%%%%%%%%%%%%%%%%%%%%%%%%%%%%%%%%%%%%%%%%%%%%%%%%%%%%%%%%%%%%%%%%%%%%%%
%                    Capitulo 2: Antecedentes                                  %
%%%%%%%%%%%%%%%%%%%%%%%%%%%%%%%%%%%%%%%%%%%%%%%%%%%%%%%%%%%%%%%%%%%%%%%%%%%%%%%%

\chapter{Antecedentes}

\section{Trabajos relacionados}

Hasta donde sabemos, no se ha desarrollado una arquitectura de planificación para
controlar el comportamiento deliberativo de un agente en \texttt{GVGAI}. No
obstante, se han desarrollado otras arquitecturas de planificación para otros juegos,
como es el caso de StarCraft. Algunas de estas arquitecturas son, por ejemplo,
\texttt{Darmok} \cite{10.1007/978-3-540-74141-1_12}, basada en casos;
\texttt{UalbertBot} \cite{Churchill2011BuildOO}, basada en la búsqueda heurística de
planes concurrentes; \texttt{EISBot} \cite{Weber2011BuildingHA}, basada en planificación reactiva;
y una arquitectura basada en \texttt{PELEA} \cite{Alczar2010peleaP} que combina
\textit{goal reasoning} \cite{Aha_2018} con planificación clásica.

A diferencia de las anteriores, nuestra arquitectura está diseñada para resolver
problemas en varios dominios de planificación y juegos en vez de solo en un
dominio de planificación y un juego, que en este caso sería StarCraft.
Por esa razón, presentamos una metodología semiautomática para la integración de un
planificador en cualquier juego de \texttt{GVGAI} que pueda ser resuelto mediante
planificación.

\section{Planificación automática}

La planificación automática es un área de la IA que se encarga del estudio de
técnicas que permiten resolver problemas en los que, dado un objetivo determinado,
se debe buscar una serie de acciones que permitan alcanzarlo \cite{10.5555/3073924}.

Para resolver problemas de planificación se requieren tres elementos:

\begin{itemize}[label=\textbullet]
    \item Un \textbf{dominio de planificación}, el cual representa un entorno y las acciones
    que puede realizar un actuador en ese entorno, además de cómo dichas acciones afectan
    al entorno.
    \item Un \textbf{problema de planificación}, en el cual representa un estado inicial
    concreto de un entorno y un objetivo que debe ser alcanzado por el actuador.
    \item Un \textbf{planificador}, que es un programa que recibe como entradas un dominio y un
    problema de planificación y devuelve un conjunto de acciones ordenadas que debe realizar
    el agente para alcanzar el objetivo desde el estado inicial descrito en el problema de
    planificación.
\end{itemize}

El lenguaje que principalmente se utiliza en planificación es \texttt{PDDL}
(\textit{Planning Domain Definition Language}). Este lenguaje surgió a finales de los 90
como un intento de estandarizar los lenguajes que se utilizan para describir los dominios
y problemas de planificación y, a medida que fueron pasando los años, fue evolucionando, añadiendo
nuevas características.


\begin{lstlisting}[language=PDDL,caption={Ejemplo de dominio \texttt{PDDL}.},captionpos=b]
(define (domain COINS)
  (:requirements :strips :typing :fluents)
  (:types
    Coin Jar
  )
  (:predicates
    (in ?c - Coin ?j - Jar)
    (on-floor ?c - Coin)
    (taken ?c - Coin)
    (empty-hand)
  )
  (:functions
    (coins-in-jar ?j)
  )
  (:action pick
     :parameters (?c - Coin)
     :precondition (and
       (empty-hand)
       (on-floor ?c)
     )
     :effect (and
       (not (empty-hand))
       (not (on-floor ?c))
       (taken ?c)
     )
  )
  (:action store
     :parameters (?c - Coin ?j - Jar)
     :precondition (taken ?c)
     :effect (and
       (not (taken ?c))
       (in ?c ?j)
       (empty-hand)
       (increase (coins-in-jar ?j) 1)
     )
  )
)
\end{lstlisting}

\begin{lstlisting}[language=PDDL,caption={Ejemplo de problema \texttt{PDDL}.},captionpos=b]
(define (problem CoinProblem)
  (:domain COINS)
  (:objects
    c1 c2 c3 - Coin
    j1 - Jar
  )
  (:init
    (empty-hand)
    (on-floor c1)
    (on-floor c2)
    (on-floor c3)
    (= (coins-in-jar j1) 0)
  )
  (:goal
    (AND
      (in c1 j1)
      (= (coins-in-jar j1) 2)
    )
  )
)
\end{lstlisting}

%% COMENTAR ESTO
%Para representar los dominios y los problemas de planificación se utiliza la lógica
%de predicados. 

\section{\texttt{GVGAI}}

\texttt{GVGAI}

\subsection{\texttt{VGDL}}