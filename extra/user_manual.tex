\documentclass[11pt,a4paper]{article}
\usepackage[spanish,es-nodecimaldot]{babel}	% Utilizar español
\usepackage[utf8]{inputenc}					% Caracteres UTF-8
\usepackage{graphicx}						% Imagenes
\usepackage[hidelinks]{hyperref}			% Poner enlaces sin marcarlos en rojo
\usepackage{fancyhdr}						% Modificar encabezados y pies de pagina
\usepackage{float}							% Insertar figuras
\usepackage[textwidth=390pt]{geometry}		% Anchura de la pagina
\usepackage[nottoc]{tocbibind}				% Referencias (no incluir num pagina indice en Indice)
\usepackage{enumitem}						% Permitir enumerate con distintos simbolos
\usepackage[T1]{fontenc}					% Usar textsc en sections
\usepackage{amsmath}						% Símbolos matemáticos
\usepackage{forest}

\definecolor{folderbg}{RGB}{124,166,198}
\definecolor{folderborder}{RGB}{110,144,169}

\def\Size{4pt}
\tikzset{
      folder/.pic={
        \filldraw[draw=folderborder,top color=folderbg!50,bottom color=folderbg]
          (-1.05*\Size,0.2\Size+5pt) rectangle ++(.75*\Size,-0.2\Size-5pt);  
        \filldraw[draw=folderborder,top color=folderbg!50,bottom color=folderbg]
          (-1.15*\Size,-\Size) rectangle (1.15*\Size,\Size);
      }
    }

% Comando para poner el nombre de la asignatura
\newcommand{\asignatura}{Trabajo Fin de Grado}
\newcommand{\autor}{Vladislav Nikolov Vasilev}
\newcommand{\titulo}{Desarrollo de una arquitectura reactiva y deliberativa
usando planificación en el entorno de juegos GVGAI}
\newcommand{\subtitulo}{Anexo - Manual de usuario}
\newcommand{\director}{Juan Fernández Olivares}

% Configuracion de encabezados y pies de pagina
\pagestyle{fancy}
\lhead{\autor{}}
\rhead{\asignatura{}}
\lfoot{Grado en Ingeniería Informática}
\cfoot{}
\rfoot{\thepage}
\renewcommand{\headrulewidth}{0.4pt}		% Linea cabeza de pagina
\renewcommand{\footrulewidth}{0.4pt}		% Linea pie de pagina

\begin{document}
\pagenumbering{gobble}

% Pagina de titulo
\begin{titlepage}

\begin{minipage}{\textwidth}

\centering

%\includegraphics[scale=0.5]{img/ugr.png}\\
\includegraphics[scale=0.3]{img/logo_ugr.jpg}\\[1cm]

\textsc{\Large \asignatura{}\\[0.2cm]}
\textsc{GRADO EN INGENIERÍA INFORMÁTICA}\\[1cm]

\noindent\rule[-1ex]{\textwidth}{1pt}\\[1.5ex]
\textbf{{\Large \titulo\\[0.5ex]}}
\noindent\rule[-1ex]{\textwidth}{3pt}\\[3.5ex]
\textsc{{\Large \subtitulo\\[0.5ex]}}

\end{minipage}

%\vspace{0.5cm}
\vspace{0.7cm}

\begin{minipage}{\textwidth}

\centering

\textbf{Autor}\\ {\autor{}}\\[2.5ex]
\textbf{Director}\\ {\director}\\[2.5ex]
\vspace{0.3cm}

\includegraphics[scale=0.3]{img/etsiit.jpeg}

\vspace{0.7cm}
\textsc{Escuela Técnica Superior de Ingenierías Informática y de Telecomunicación}\\
\vspace{1cm}
\textsc{Curso 2019-2020}
\end{minipage}
\end{titlepage}

\pagenumbering{arabic}
\tableofcontents
\thispagestyle{empty}				% No usar estilo en la pagina de indice

\newpage

\setlength{\parskip}{1em}

\section{Instalación}

El proyecto se encuentra alojado en un repositorio público de \texttt{GitHub}. Este repositorio
encontrarse en el siguiente enlace: \url{https://github.com/Vol0kin/gvgai-pddl}.

Las dependencias del sistema y el proceso de instalación se encuentran descritos en la página principal.
En esta página también puede encontrarse otra información, como por ejemplo cómo generar plantillas
de archivos de configuración, cómo ejecutar el sistema y cómo se puede ejecutar el planificador de
\texttt{Planning.Domains} en local.

\section{Estructura del proyecto}

La estructura del proyecto puede verse en la figura \ref{fig:project-structure}. Ahí pueden verse los elementos
más destacados, tanto directorios como archivos:

\begin{itemize}[label=\textbullet]
    \item El directorio \texttt{config} contiene algunos ejemplos de  archivos de configuración en formato
    \texttt{YAML}. No es obligatorio que los archivos de configuración se generen en este directorio, pero
    sí que sería una buena idea tener un directorio específico donde agruparlos.
    \item El directorio \texttt{docs} contiene los \texttt{JavaDoc} que se han generado para el código
    fuente del sistema.
    \item El directorio \texttt{domains} contiene algunos ejemplos de dominios \texttt{PDDL} que se han
    creado para algunos de los juegos.
    \item El directorio \texttt{examples} es uno de los directorios originales de \texttt{GVGAI}. Aquí
    se incluyen los archivos en formato \texttt{VGDL} que describen los juegos y los niveles. Dentro de
    este directorio se encuentra el directorio \texttt{gridphysics}, el cual contiene todos los juegos
    que puede ejecutar el sistema.
    \item El directorio \texttt{sprites} es otro de los directorios originales de \texttt{GVGAI}. Aquí se
    encuentran los \textit{sprites} de los elementos de los juegos.
    \item El directorio \texttt{src} contiene el código fuente del sistema, el código de los tests y los
    recuros que utilizan los tests (dominios, problemas y archivos de configuración). La
    estructura de este directorio sigue la estructura típica de un proyecto de Maven. El directorio
    \texttt{src/main/java} contiene el código fuente del sistema. Dentro de dicho directorio se encuentra
    el directorio \texttt{controller}, el cual contiene el código fuente implementado.
    
    \begin{figure}[H]

    \begin{forest}
      for tree={
        font=\ttfamily,
        grow'=0,
        child anchor=west,
        parent anchor=south,
        anchor=west,
        calign=first,
        inner xsep=7pt,
        edge path={
          \noexpand\path [draw, \forestoption{edge}]
          (!u.south west) +(7.5pt,0) |- (.child anchor) pic {folder} \forestoption{edge label};
        },
        % style for your file node 
        file/.style={edge path={\noexpand\path [draw, \forestoption{edge}]
          (!u.south west) +(7.5pt,0) |- (.child anchor) \forestoption{edge label};},
          inner xsep=2pt,font=\small\ttfamily
                     },
        before typesetting nodes={
          if n=1
            {insert before={[,phantom]}}
            {}
        },
        fit=band,
        before computing xy={l=15pt},
      }
    [gvgai-pddl
      [config
      ]
      [docs
      ]
      [domains
      ]
      [examples
      ]
      [sprites
      ]
      [src
        [main
          [java
            [controller]
            [core]
            [ontology]
            [testing]
            [tools]
            [tracks]
          ]
        ]
        [test
          [java
            [controller
            ]
          ]
          [resources]
        ]
      ]
      [test\_files]
      [pom.xml,file]
      [requirements.txt,file]
      [gen\_config\_file.py,file]
    ]
    \end{forest}
    \caption{Estructura del proyecto.}
    \label{fig:project-structure}
\end{figure}

    \item El archivo \texttt{pom.xml} contiene información sobre el proyecto de Maven, como por ejemplo
    las dependencias. También contiene la configuración del proyecto, como por ejemplo las tareas que
    se tienen realizar en algunos de los pasos de la construcción del proyecto. Se recomienda no
    modificar este archivo a menos que se sepa muy bien qué es lo que se está modificando.
    \item El archivo \texttt{requirements.txt} contiene las dependencias del \textit{script} de \texttt{Python}
    para la generación de plantillas de archivos de configuración.
    \item Por último, el \textit{script} \texttt{gen\_config\_file.py} es el que permite generar plantillas de
    archivos de configuración.
\end{itemize}

Los archivos que se encuentran en el directorio \texttt{controller} se pueden ver en la figura \ref{fig:controller}.
En estos archivos se implementan las siguientes funcionalidades:

\begin{itemize}[label=\textbullet]
    \item En el archivo \texttt{Agenda.java} se implementa la clase \texttt{Agenda}, la cual permite la gestión
    de los objetivos de forma sencilla.
    \item El fichero \texttt{GameInformation.java} contiene la clase \texttt{GameInformation}, la cual es la
    responsable de guardar la información que se carga de los archivos de configuración.
    \item En el archivo \texttt{PDDLAction.java} se encuentra implementada la clase \texttt{PDDLAction},
    la cual representa una acción \texttt{PDDL}. También se encuentra implementada la clase \texttt{PDDLEffect}
    como clase anidada a la anterior. Esta clase representa un efecto de una acción \texttt{PDDL}.
    \begin{figure}[H]
\begin{forest}
      for tree={
        font=\ttfamily,
        grow'=0,
        child anchor=west,
        parent anchor=south,
        anchor=west,
        calign=first,
        inner xsep=7pt,
        edge path={
          \noexpand\path [draw, \forestoption{edge}]
          (!u.south west) +(7.5pt,0) |- (.child anchor) pic {folder} \forestoption{edge label};
        },
        % style for your file node 
        file/.style={edge path={\noexpand\path [draw, \forestoption{edge}]
          (!u.south west) +(7.5pt,0) |- (.child anchor) \forestoption{edge label};},
          inner xsep=2pt,font=\small\ttfamily
                     },
        before typesetting nodes={
          if n=1
            {insert before={[,phantom]}}
            {}
        },
        fit=band,
        before computing xy={l=15pt},
      }
    [src
        [main
          [java
            [controller
              [Agenda.java,file]
              [GameInformation.java,file]
              [PDDLAction.java,file]
              [PDDLPlan.java,file]
              [PDDLSingleGoal.java,file]
              [PlannerException.java,file]
              [PlanningAgent.java,file]
              [Position.java,file]
            ]
           ]
         ]
    ]
\end{forest}
\caption{Contenido del directorio \texttt{controller}.}
\label{fig:controller}
\end{figure}

    \item En el archivo \texttt{PDDLPlan.java} se implementa la clase \texttt{PDDLPlan}, la cual representa un
    plan de \texttt{PDDL}.
    \item El archivo \texttt{PDDLSingleGoal.java} contiene la implementación de la clase \texttt{PDDLSingleGoal},
    la cual representa un objetivo \texttt{PDDL}.
    \item En \texttt{PlannerException.java} se tiene la excepción que se lanza cuando se produce algún error al
    llamar al planificador en la nube.
    \item En el archivo \texttt{PlanningAgent.java} se encuentra la clase \texttt{PlanningAgent}, la cual
    contiene el agente implementado.
    \item El archivo \texttt{Position.java} contiene un enumerado que indica cuatro posibles posiciones:
    \texttt{UP}, \texttt{DOWN}, \texttt{LEFT}, \texttt{RIGHT}. Estas posiciones se utilizan en algunas
    de las clases anteriores.
\end{itemize}

Por último, en la figura \ref{fig:tests} se observa la estructura del directorio que contiene los tests
que se han creado. Estos tests se hacen sobre la funcionalidad pública que exponen las correspondientes clases.
Como se mencionó anteriormente, se utilizan los recursos localizados en \texttt{src/test/resources} para
llevarlos a cabo.

\begin{figure}[H]
\begin{forest}
      for tree={
        font=\ttfamily,
        grow'=0,
        child anchor=west,
        parent anchor=south,
        anchor=west,
        calign=first,
        inner xsep=7pt,
        edge path={
          \noexpand\path [draw, \forestoption{edge}]
          (!u.south west) +(7.5pt,0) |- (.child anchor) pic {folder} \forestoption{edge label};
        },
        % style for your file node 
        file/.style={edge path={\noexpand\path [draw, \forestoption{edge}]
          (!u.south west) +(7.5pt,0) |- (.child anchor) \forestoption{edge label};},
          inner xsep=2pt,font=\small\ttfamily
                     },
        before typesetting nodes={
          if n=1
            {insert before={[,phantom]}}
            {}
        },
        fit=band,
        before computing xy={l=15pt},
      }
    [src
        [test
          [java
            [controller
              [TestAgenda.java,file]
              [TestPDDLAction.java,file]
              [TestPDDLPlan.java,file]
              [TestPDDLSingleGoal.java,file]
              [TestPlanningAgent.java,file]
            ]
           ]
         ]
    ]
\end{forest}
\caption{Estructura del directorio que contiene los tests para los archivos del paquete \texttt{controller}.}
\label{fig:tests}
\end{figure}

\section{Documentación detallada del sistema}

En la página principal del proyecto de \texttt{GitHub} se puede encontrar un enlace a la documentación detallada
de todo el código fuente, incluyendo la del sistema.

\newpage

\end{document}
