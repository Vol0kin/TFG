\documentclass[11pt,a4paper]{book}
\usepackage[spanish,es-nodecimaldot]{babel}	% Utilizar español
\usepackage[utf8]{inputenc}					% Caracteres UTF-8
\usepackage{multirow}                       % Multifila en tablas
\usepackage{lscape}
\usepackage{graphicx}						% Imagenes
\usepackage[hidelinks]{hyperref}			% Poner enlaces sin marcarlos en rojo
\usepackage{fancyhdr}						% Modificar encabezados y pies de pagina
\usepackage{float}							% Insertar figuras
\usepackage[textwidth=390pt]{geometry}		% Anchura de la pagina
\usepackage[nottoc]{tocbibind}				% Referencias (no incluir num pagina indice en Indice)
\usepackage{enumitem}						% Permitir enumerate con distintos simbolos
\usepackage[T1]{fontenc}					% Usar textsc en sections
\usepackage{amsmath}						% Símbolos matemáticos
\usepackage{emptypage}                      % Dejar sin estilo las paginas vacias
\usepackage[Sonny]{fncychap}                % Capitulos con encabezado chulo :D

\usepackage{tikz}
\usetikzlibrary{positioning, arrows, shapes, automata}

% Ajustar tikzet
\tikzset{Node style/.style={thick, draw=black, circle, align=center, minimum width=70pt}}
\tikzset{
	->, % Hace que los arcos sean dirigidos
	>=stealth, % Hace que la punta de las flechas sean gruesas
	node distance=7cm, % Distancia minima entre nodos
}

\usepackage[
    backend=biber,
    style=numeric,
    sorting=ynt
]{biblatex}                                 % Gestion de bibliografia
\addbibresource{bibliography.bib}           % Archivo que contiene la bibliografia

\usepackage{xcolor}
\definecolor{backcolour}{rgb}{0.95,0.95,0.92}
\usepackage{listings}

\newcommand\YAMLcolonstyle{\color{red}\mdseries}
\newcommand\YAMLkeystyle{\color{black}\bfseries}
\newcommand\YAMLvaluestyle{\color{blue}\mdseries}

\makeatletter

% here is a macro expanding to the name of the language
% (handy if you decide to change it further down the road)
\newcommand\language@yaml{yaml}

\expandafter\expandafter\expandafter\lstdefinelanguage
\expandafter{\language@yaml}
{
  backgroundcolor=\color{backcolour},
  keywords={true,false,null,y,n},
  keywordstyle=\color{darkgray}\bfseries,
  basicstyle=\YAMLkeystyle,                                 % assuming a key comes first
  sensitive=false,
  comment=[l]{\#},
  morecomment=[s]{/*}{*/},
  commentstyle=\color{purple}\ttfamily,
  stringstyle=\YAMLvaluestyle\ttfamily,
  moredelim=[l][\color{orange}]{\&},
  moredelim=[l][\color{magenta}]{*},
  moredelim=**[il][\YAMLcolonstyle{:}\YAMLvaluestyle]{:},   % switch to value style at :
  morestring=[b]',
  morestring=[b]",
  literate =    {>}{{\textcolor{red}\textgreater}}1     
                {|}{{\textcolor{red}\textbar}}1 
                {\ -\ }{{\mdseries\ -\ }}3,
}

% switch to key style at EOL
\lst@AddToHook{EveryLine}{\ifx\lst@language\language@yaml\YAMLkeystyle\fi}
\makeatother

%%%%%%%%%%%%%%%%%%%%%%%%%%%%%%%%%%%%%%%%%%%%%%%%%%%%%%%%%%%%%%%%%%%%%%%%%%%%%%%%
% Definir comandos
\newcommand{\asignatura}{Trabajo Fin de Grado}
\newcommand{\autor}{Vladislav Nikolov Vasilev}
\newcommand{\titulo}{Implementación de una arquitectura reactiva y deliberativa
usando planificación en el entorno de juegos GVGAI}
\newcommand{\subtitulo}{Subtitulo}
\newcommand{\director}{Juan Fernández Olivares}
\newcommand{\grado}{Grado en Ingeniería Informática}

% Configuracion de encabezados y pies de pagina
\pagestyle{fancy}

% Paginas pares:
%   - HEADER: poner solo el nombre del capitulo
%   - FOOTER: num. pagina a la izquierda y titulacion
\fancyhead[RE]{}
\fancyhead[LE]{\textbf{\nouppercase{\leftmark}}}
\fancyfoot[RE]{\grado}
\fancyfoot[LE]{\thepage}

% Paginas impares: poner la seccion y el autor
\fancyhead[RO]{\textbf{\nouppercase{\small{\rightmark}}}}
\fancyhead[LO]{\autor}

\fancyfoot[RO]{\thepage}
\fancyfoot[LO]{\asignatura}

\fancyfoot[C]{}

% Poner las lineas
\renewcommand{\headrulewidth}{0.4pt}		% Linea cabeza de pagina
\renewcommand{\footrulewidth}{0.4pt}		% Linea pie de pagina

\begin{document}
\pagenumbering{gobble}

% Insertar portada
%%%%%%%%%%%%%%%%%%%%%%%%%%%%%%%%%%%%%%%%%%%%%%%%%%%%%%%%%%%%%%%%%%%%%%%%%%%%%%%%
%                           Portada del proyecto                               %
%%%%%%%%%%%%%%%%%%%%%%%%%%%%%%%%%%%%%%%%%%%%%%%%%%%%%%%%%%%%%%%%%%%%%%%%%%%%%%%%
\begin{titlepage}

\begin{minipage}{\textwidth}

\centering

%\includegraphics[scale=0.5]{img/ugr.png}\\
\includegraphics[scale=0.3]{img/logo_ugr.jpg}\\[1cm]

\textsc{\Large \asignatura{}\\[0.2cm]}
\textsc{GRADO EN INGENIERÍA INFORMÁTICA}\\[1cm]

\noindent\rule[-1ex]{\textwidth}{1pt}\\[1.5ex]
\textbf{{\Large \titulo\\[0.5ex]}}
%\textsc{{\Large \subtitulo\\}}
\noindent\rule[-1ex]{\textwidth}{3pt}\\[3.5ex]

\end{minipage}

%\vspace{0.5cm}
\vspace{0.8cm}

\begin{minipage}{\textwidth}

\centering

\textbf{Autor}\\ {\autor{}}\\[2.5ex]
\textbf{Tutor}\\ {\tutor}\\[2.5ex]
\vspace{0.3cm}

\includegraphics[scale=0.3]{img/etsiit.jpeg}

\vspace{0.7cm}
\textsc{Escuela Técnica Superior de Ingenierías Informática y de Telecomunicación}\\
\vspace{1cm}
\textsc{Granada, Julio de 2020}
\end{minipage}
\end{titlepage}

% Insertar prefacio
%\thispagestyle{empty}
%\cleardoublepage

%\thispagestyle{empty}


\cleardoublepage
\thispagestyle{empty}

\begin{center}
{\large\bfseries Título del Proyecto: Subtítulo del proyecto}\\
\end{center}
\begin{center}
\autor{}\\
\end{center}

%\vspace{0.7cm}
\noindent{\textbf{Palabras clave}: palabra\_clave1, palabra\_clave2, palabra\_clave3, ......}\\

\vspace{0.7cm}
\noindent{\textbf{Resumen}}\\

Poner aquí el resumen.
\cleardoublepage


\thispagestyle{empty}


\begin{center}
{\large\bfseries Project Title: Project Subtitle}\\
\end{center}
\begin{center}
\autor{}\\
\end{center}

%\vspace{0.7cm}
\noindent{\textbf{Keywords}: Keyword1, Keyword2, Keyword3, ....}\\

\vspace{0.7cm}
\noindent{\textbf{Abstract}}\\

Write here the abstract in English.

\cleardoublepage
\thispagestyle{empty}

\noindent\rule[-1ex]{\textwidth}{2pt}\\[4.5ex]

Yo, \textbf{\autor{}}, alumno de la titulación Ingeniería Informática de la \textbf{Escuela Técnica Superior
de Ingenierías Informática y de Telecomunicación de la Universidad de Granada}, con NIE X8743846M, autorizo la
ubicación de la siguiente copia de mi Trabajo Fin de Grado en la biblioteca del centro para que pueda ser
consultada por las personas que lo deseen.

\vspace{6cm}

\noindent Fdo: \autor{}

\vspace{2cm}

\begin{flushright}
Granada a X de mes de 201 .
\end{flushright}


\cleardoublepage
\thispagestyle{empty}

\noindent\rule[-1ex]{\textwidth}{2pt}\\[4.5ex]

D. \textbf{\director{}}, Profesor del Departamento de Ciencias de la Computación e Inteligencia Artificial
de la Universidad de Granada.


\vspace{0.5cm}

\textbf{Informa:}

\vspace{0.5cm}

Que el presente trabajo, titulado \textit{\textbf{Título del proyecto, Subtítulo del proyecto}},
ha sido realizado bajo su supervisión por \textbf{\autor{}}, y autorizo la defensa de dicho trabajo ante el tribunal
que corresponda.

\vspace{0.5cm}

Y para que conste, expide y firma el presente informe en Granada a X de mes de 201 .

\vspace{1cm}

\textbf{El director:}

\vspace{5cm}

\noindent \textbf{\director{} }

\chapter*{Agradecimientos}
\thispagestyle{empty}

       \vspace{1cm}


Poner aquí agradecimientos...



% Indice
\thispagestyle{empty}
\tableofcontents

% Lista de figuras
\thispagestyle{empty}
\listoffigures

%\thispagestyle{empty}
%\lstlistoflistings

\newpage

% Comenzar la numeracion de las paginas a partir del primer capitulo
\pagenumbering{arabic}
\setlength{\parskip}{1em}

% Insertar capitulo 1: Introduccion
%%%%%%%%%%%%%%%%%%%%%%%%%%%%%%%%%%%%%%%%%%%%%%%%%%%%%%%%%%%%%%%%%%%%%%%%%%%%%%%%
%                    Capitulo 1: Introduccion                                  %
%%%%%%%%%%%%%%%%%%%%%%%%%%%%%%%%%%%%%%%%%%%%%%%%%%%%%%%%%%%%%%%%%%%%%%%%%%%%%%%%

\chapter{Introducción}

La planificación automática es una rama de la Inteligencia Artificial cuyo objetivo
último es proporcionar planes que permitan resolver problemas \cite{10.5555/3073924}.
Para ello se hace uso de un \textbf{\textit{planificador}}, que es un programa que recibe
como entradas un \textbf{\textit{dominio de planificación}}, es decir, una representación
del mundo, y un \textbf{\textit{problema de planificación}}, que es un estado concreto de
ese mundo que debe ser resuelto. El resultado obtenido es una secuencia ordenada de acciones
que permiten alcanzar el objetivo descrito en el problema desde el estado
inicial que se describe en él.

A pesar de que la planificación automática ha sido integrada con éxito en muchas aplicaciones
reales como por ejemplo la robótica o los vehículos no tripulados, el campo de los videojuegos
sigue suponiéndole un gran reto. Esto se debe a que los videojuegos presentan, en líneas generales,
entornos dinámicos y complejos, los cuales requieren una respuesta rápida ante los cambios que se producen
y presentan espacios de búsqueda muy grandes debido al elevado número de acciones que existen.
El proceso de planificación es costoso y lento, por lo que los planificadores no se ven
capaces de ofrecer una respuesta rápida a los cambios que se producen en su entorno.

El reto que supone integrar una arquitectura deliberativa basada en planificación en un videojuego ha
abierto muchas líneas de investigación. La mayoría de ellas se centran en torno al juego
StarCraft \cite{10.1007/978-3-540-74141-1_12, Churchill2011BuildOO, Weber2011BuildingHA, Aha_2018}
debido a la enorme dinamicidad y complejidad que se puede encontrar en dicho juego.
Existen, no obstante, otros juegos que se podrían utilizar para dichas pruebas.
\texttt{GVGAI} (\textit{General Video Game AI})\cite{7038214} es un entorno que
dispone de más de cien juegos diferentes. Entre ellos existen juegos que pueden ser resueltos
muy fácilmente mediante técnicas de planificación, y también hay otros juegos con entornos
dinámicos que pueden llegar a suponer un auténtico reto para los planificadores.

Si bien es cierto que \texttt{GVGAI} tiene una gran cantidad de juegos que pueden resultar
interesantes desde el punto de vista de la planificación, hay que tener en cuenta que el entorno
no fue concebido para la creación de arquitecturas deliberativas basadas en planificación.
Este entorno surgió para la creación y competición de arquitecturas reactivas basadas en técnicas como el
Aprendizaje por Refuerzo o MCTS (\textit{Monte Carlo Tree Search}). Por tanto, crear agentes basados en
planificación para los distintos juegos de \texttt{GVGAI} puede llegar a ser un proceso costoso y
no tan directo como crear un agente reactivo. Es en este contexto que surge el presente trabajo.

El objetivo principal de este trabajo es proporcionar una arquitectura reactiva y deliberativa basada
en planificación en el entorno de juegos \texttt{GVGAI}. Además, a diferencia de las anteriores arquitecturas,
las cuales estaban creadas específicamente para un único juego, la propuesta que se hace en este trabajo
tiene también como objetivo ser lo suficientemente general como para permitir resolver cualquier juego
del entorno que se desee mediante planificación, siempre y cuando éste pueda ser resuelto utilizando este tipo
de técnicas.

Esta propuesta puede llegar a ser bastante útil por dos motivos principales. Por un
lado abre nuevas vías para experimentar con técnicas de planificación en \texttt{GVGAI}, lo cual es de
sumo interés para el desarrollo de arquitecturas \textit{online} que integran un planificador que
dirige el comportamiento deliberativo de un agente. Por otro lado, puede utilizarse como herramienta
docente para enseñar cómo se representan los problemas de planificación y para aprender más sobre la propia
planificación.

El trabajo se ha estructurado de la siguiente manera. En el capítulo 2 se hará
una introducción, comentando los trabajos relacionados e introduciendo conceptos
fundamentales como la planificación y \texttt{GVGAI}. Después, en el capítulo 3 se
detallará el plan de trabajo seguido, esto es, la metodología de trabajo y la
temporización. Una vez hecho esto, en el capítulo 4 se presentará la arquitectura
general del sistema, y seguidamente, en el capítulo 5 se comentará la arquitectura
en más detalle. En el capítulo 6 se hablará de la creación y del formato del archivo
de configuración. Una vez hecha la explicación pasaremos al capítulo 7,
donde se hará una descripción general de la implementación \textit{software} que se
ha llevado a cabo. En el capítulo 8 se presentarán una serie de experimentos que
se han realizado con el sistema. Por último, en el capítulo 9 se presentarán las
conclusiones a las que se han llegado y trabajos futuros.

%%%%%%%%%%%%%%%%%%%%%%%%%%%%%%%%%%%%%%%%%%%%%%%%%%%%%%%%%%%%%%%%%%%%%%%%%%%%%%%%
%                    Capitulo 2: Antecedentes                                  %
%%%%%%%%%%%%%%%%%%%%%%%%%%%%%%%%%%%%%%%%%%%%%%%%%%%%%%%%%%%%%%%%%%%%%%%%%%%%%%%%

\chapter{Antecedentes}

\section{Trabajos relacionados}

Hasta donde sabemos, no se ha desarrollado una arquitectura de planificación para
controlar el comportamiento deliberativo de un agente en \texttt{GVGAI}. No
obstante, se han desarrollado otras arquitecturas de planificación para otros juegos,
como es el caso de StarCraft. Algunas de estas arquitecturas son, por ejemplo,
\texttt{Darmok} \cite{10.1007/978-3-540-74141-1_12}, basada en casos;
\texttt{UalbertBot} \cite{Churchill2011BuildOO}, basada en la búsqueda heurística de
planes concurrentes; \texttt{EISBot} \cite{Weber2011BuildingHA}, basada en planificación reactiva;
y una arquitectura basada en \texttt{PELEA} \cite{Alczar2010peleaP} que combina
\textit{goal reasoning} \cite{Aha_2018} con planificación clásica.

A diferencia de las anteriores, nuestra arquitectura está diseñada para resolver
problemas en varios dominios de planificación y juegos en vez de solo en un
dominio de planificación y un juego, que en este caso sería StarCraft.
Por esa razón, presentamos una metodología semiautomática para la integración de un
planificador en cualquier juego de \texttt{GVGAI} que pueda ser resuelto mediante
planificación.

\section{Planificación automática}

La planificación automática


\section{\texttt{GVGAI}}

\texttt{GVGAI}

\subsection{\texttt{VGDL}}

%%%%%%%%%%%%%%%%%%%%%%%%%%%%%%%%%%%%%%%%%%%%%%%%%%%%%%%%%%%%%%%%%%%%%%%%%%%%%%%%
%                    Capitulo 3: Plan de trabajo                               %
%%%%%%%%%%%%%%%%%%%%%%%%%%%%%%%%%%%%%%%%%%%%%%%%%%%%%%%%%%%%%%%%%%%%%%%%%%%%%%%%

\chapter{Plan de trabajo}


%%%%%%%%%%%%%%%%%%%%%%%%%%%%%%%%%%%%%%%%%%%%%%%%%%%%%%%%%%%%%%%%%%%%%%%%%%%%%%%%
%              Capitulo 4: Arquitectura general del sistema                    %
%%%%%%%%%%%%%%%%%%%%%%%%%%%%%%%%%%%%%%%%%%%%%%%%%%%%%%%%%%%%%%%%%%%%%%%%%%%%%%%%

\chapter{Arquitectura general del sistema}

El sistema está estructurado en una serie de módulos lógicos que representan una
tarea de alto nivel. Un módulo lógico se compone a su vez de diversos módulos
funcionales o procesos, los cuáles llevan a cabo una tarea concreta y se comunican con
otros módulos funcionales del mismo módulo lógico para llevar a cabo dicha tarea de alto
nivel.

La tarea que un módulo funcional lleva a cabo puede realizarse de forma automática o
puede llegar a requerir de la interacción con el usuario para poder ser completada.
Cabe destacar además que, en determinados casos, los módulos funcionales de un módulo
lógico pueden comunicarse con los módulos funcionales de otros módulos lógicos, lo cuál se
describirá más detenidamente más adelante.

Por lo pronto, los módulos lógicos principales que se han considerado a la hora de definir
la arquitectura son los siguientes:

\begin{itemize}[label=\textbullet]
    \item Un módulo de interacción con el usuario (\textit{User interaction}).
    \item Un módulo de planificación (\textit{Planning}).
    \item Un módulo de monitorización y ejecución del plan (\textit{Plan Execution \& Monitoring}).
\end{itemize}

En la figura \ref{fig:system_arch} pueden verse los distintos módulos lógicos y funcionales que
componen el sistema, además de cómo están organizados, cómo se comunican y qué información se
envían entre ellos.

\begin{figure}[H]
    \centering
    \includegraphics[scale=0.4]{img/CH04/system_arch.png}
    \caption{Esquema de la arquitectura general del sistema.}
    \label{fig:system_arch}
\end{figure}

Las distintas figuras, colores y flechas que se han utilizado a la hora de construir el
esquema representan lo siguiente:

\begin{itemize}[label=\textbullet]
    \item Los cilindros azules representan ficheros que se utilizan en el sistema.
    \item Los círculos de color morado representan módulos funcionales que llevan a cabo
    tareas de traducción.
    \item Los cuadrados representan módulos funcionales que realizan distintos tipos de
    tareas. Según el color que tengan, las tareas llevadas a cabo tendrán un mayor o menor
    grado de automatización:
    \begin{itemize}[label=\textendash]
        \item El color rojo se utiliza en tareas que se llevan a cabo automáticamente.
        \item El color verde se utiliza en tareas que requieren de la interacción del usuario
        en un mayor o menor grado para poder ser completadas.
    \end{itemize}
    \item El hexágono amarillo representa el entorno de juego.
    \item La nube de color gris representa el planificador, el cuál es un servicio en la
    nube.
    \item Las flechas normales representan información que se envían distintos módulos
    funcionales entre ellos.
    \item Las flechas discontinuas representan información proveniente de los archivos.
\end{itemize}

%%%%%%%%%%%%%%%%%%%%%%%%%%%%%%%%%%%%%%%%%%%%%%%%%%%%%%%%%%%%%%%%%%%%%%%%%%%%%%%%
%              Sección 4.1: Descripción general de la arquitectura             %
%%%%%%%%%%%%%%%%%%%%%%%%%%%%%%%%%%%%%%%%%%%%%%%%%%%%%%%%%%%%%%%%%%%%%%%%%%%%%%%%

\section{Descripción general de la arquitectura}

Visto el esquema de la arquitectura general, vamos a comentar brevemente cuál es
la funcionalidad básica de cada módulo lógico y cómo interactúan entre ellos.
En el próximo capítulo desglosaremos cada uno de ellos y estudiaremos los módulos
funcionales que los componen, de forma que se obtendrá una mejor visión de cómo
funciona cada componente y de cómo interactúa con las demás.

Empecemos por la parte fundamental, la cuál es el \textbf{módulo de interacción con el usuario}.
Como se puede suponer y por lo que se ha comentado anteriormente, este es el módulo
menos automatizado, ya que es con el que interactúa directamente el usuario. Es aquí
donde se encuentran dos de los procesos más importantes: la creación del dominio y la
creación del archivo de configuración que utilizará el sistema.

Por una parte, el usuario tiene que definir el dominio de planificación, que es el que se
utilizará en el sistema para representar el estado del juego en formato \texttt{PDDL}.
Por otra parte, el usuario debe especificar una serie de parámetros en el archivo de
configuración, los cuáles se utilizarán en el sistema para, entre otras cosas, traducir los
estados de observación del juego a predicados \texttt{PDDL}.

El siguiente módulo lógico que vamos a comentar es el \textbf{módulo de planificación}.
Este módulo se encarga de gestionar los objetivos cuando sea necesario, estableciendo el
objetivo actual y determinando cuáles se han cumplido y cuáles no. También es el responsable
de obtener un plan válido hasta un predicado objetivo dado, y de traducir posteriormente
dicho plan para que pueda ser ejecutado en el entorno de juego.

Por último tenemos el \textbf{módulo de ejecución del plan y de monitorización}. Como su
propio nombre indica, este módulo se encarga de ejecutar el plan obtenido por el módulo
de planificación y de monitorizar dicha ejecución, determinando en el proceso si se ha
producido alguna discrepancia, y si por tanto hace falta replanificar. En el proceso
tiene que realizar la traducción del estado de observación del juego a predicados
\texttt{PDDL}, ya que esta información se utiliza por el monitor para estudiar la
existencia de discrepancias.

En cuanto a las interacciones entre los módulos, el módulo de ejecución del plan y de
monitorización y el de planificación interactúan directamente entre ellos. El primero
comunica al segundo si se ha producido alguna discrepancia, y el segundo debe determinar
un nuevo objetivo y encontrar un plan hasta dicho objetivo. Posteriormente, el plan obtenido
es comunicado al primer módulo, el cuál se encargará de hacer las operaciones pertinentes
con él.

El módulo de interacción con el usuario se comporta como un proveedor con los otros dos
módulos, ya que les proporciona la información necesaria para que éstos puedan
llevar a cabo ciertas tareas. Al funcionar como un proveedor, la comunicación es unilateral,
ya que no obtiene ningún tipo de respuesta de ellos.


%%%%%%%%%%%%%%%%%%%%%%%%%%%%%%%%%%%%%%%%%%%%%%%%%%%%%%%%%%%%%%%%%%%%%%%%%%%%%%%%
%                    Capitulo 5: Descripcion detallada                         %
%%%%%%%%%%%%%%%%%%%%%%%%%%%%%%%%%%%%%%%%%%%%%%%%%%%%%%%%%%%%%%%%%%%%%%%%%%%%%%%%

\chapter{Descripción detallada}


%%%%%%%%%%%%%%%%%%%%%%%%%%%%%%%%%%%%%%%%%%%%%%%%%%%%%%%%%%%%%%%%%%%%%%%%%%%%%%%%
%               Capitulo 6: Configuracion                  %
%%%%%%%%%%%%%%%%%%%%%%%%%%%%%%%%%%%%%%%%%%%%%%%%%%%%%%%%%%%%%%%%%%%%%%%%%%%%%%%%

\chapter{Configuración}

%%%%%%%%%%%%%%%%%%%%%%%%%%%%%%%%%%%%%%%%%%%%%%%%%%%%%%%%%%%%%%%%%%%%%%%%%%%%%%%%
%                    Capitulo 7: Diseño software del sistema                   %
%%%%%%%%%%%%%%%%%%%%%%%%%%%%%%%%%%%%%%%%%%%%%%%%%%%%%%%%%%%%%%%%%%%%%%%%%%%%%%%%

\chapter{Diseño software del sistema}

\section{Implementación}

Comentar brevemente la implementación llevada a cabo.

\section{Arquitectura software del sistema}

Comentar la arquitectura software del sistema.

\begin{figure}[H]
    \centering
    \includegraphics[angle=90, origin=c, scale=0.18]{img/CH07/class_diagram.png}
    \caption{Diagrama de clases.}
    \label{fig:class_diagram}
\end{figure}

\begin{figure}[H]
    \centering
    \includegraphics[scale=0.48]{img/CH07/package_diagram.png}
    \caption{Diagrama de paquetes.}
    \label{fig:package_diagram}
\end{figure}

\section{Descripción de las clases}

\begin{figure}[H]
    \centering
    \includegraphics[scale=0.22]{img/CH07/sequence_diagram.png}
    \caption{Diagrama de secuencia del método \texttt{act()}.}
    \label{fig:sequence_diagram}
\end{figure}

%%%%%%%%%%%%%%%%%%%%%%%%%%%%%%%%%%%%%%%%%%%%%%%%%%%%%%%%%%%%%%%%%%%%%%%%%%%%%%%%
%                    Capitulo 8: Experimentacion                               %
%%%%%%%%%%%%%%%%%%%%%%%%%%%%%%%%%%%%%%%%%%%%%%%%%%%%%%%%%%%%%%%%%%%%%%%%%%%%%%%%

\chapter{Experimentación}

Una vez que el sistema ha sido implementado y se ha validado el correcto funcionamiento
de las funcionalidades desarrolladas,

\begin{table}[H]
\centering
\begin{tabular}{|c|ccc|}
\hline
\textbf{Juego} & \textbf{Predicados} & \textbf{Tipos} & \textbf{Acciones} \\ \hline
\textit{Boulderdash} & 15 & 9 & 17 \\ \hline
\textit{Ice and Fire} & 13 & 10 & 14 \\ \hline
\textit{Labyrinth Dual} & 12 & 9 & 14 \\ \hline
\end{tabular}
\caption{Información sobre los dominios creados para cada juego.}
\label{tab:info-domains}
\end{table}


% Please add the following required packages to your document preamble:
% \usepackage{multirow}
% \usepackage{graphicx}
% \usepackage{lscape}
\begin{landscape}
\begin{table}[]
\centering
\resizebox{1.5\textwidth}{!}{%
\begin{tabular}{|c|c|cccccccc|}
\hline
\textbf{Juego} &
  \textbf{Nivel} &
  \textbf{Objetivos} &
  \textbf{\begin{tabular}[c]{@{}c@{}}Predicados problema\\ inicial\end{tabular}} &
  \textbf{\begin{tabular}[c]{@{}c@{}}Objetos problema\\ inicial\end{tabular}} &
  \textbf{\begin{tabular}[c]{@{}c@{}}Tiempo medio\\ ejecución (s)\end{tabular}} &
  \textbf{\begin{tabular}[c]{@{}c@{}}Tiempo medio\\ predicados\\ conectividad (s)\end{tabular}} &
  \textbf{\begin{tabular}[c]{@{}c@{}}Tiempo medio\\ traducción estado\\ de observación (s)\end{tabular}} &
  \textbf{\begin{tabular}[c]{@{}c@{}}Tiempo medio\\ generación archivo\\ de problema (s)\end{tabular}} &
  \textbf{\begin{tabular}[c]{@{}c@{}}Tiempo medio total\\ generación problema\\ (s)\end{tabular}} \\ \hline
\multirow{5}{*}{\textit{Boulderdash}}    & \textbf{0} & 10 & 1735 & 400 & 0.4967 & 0.0313 & 0.0024 & 0.0038 & 0.0375 \\ \cline{2-10} 
                                         & \textbf{1} & 10 & 1721 & 393 & 0.428  & 0.0323 & 0.0029 & 0.0039 & 0.0392 \\ \cline{2-10} 
                                         & \textbf{2} & 10 & 1717 & 391 & 0.5293 & 0.0347 & 0.0034 & 0.0044 & 0.0425 \\ \cline{2-10} 
                                         & \textbf{3} & 10 & 1733 & 399 & 0.7487 & 0.0327 & 0.0049 & 0.0048 & 0.0423 \\ \cline{2-10} 
                                         & \textbf{4} & 10 & 1731 & 398 & 0.4403 & 0.0317 & 0.0030 & 0.0039 & 0.0386 \\ \hline
\multirow{5}{*}{\textit{Ice and Fire}}   & \textbf{0} & 3  & 1215 & 391 & 0.417  & 0.027  & 0.0026 & 0.003  & 0.0326 \\ \cline{2-10} 
                                         & \textbf{1} & 3  & 1234 & 410 & 0.4827 & 0.0307 & 0.0028 & 0.0056 & 0.0390 \\ \cline{2-10} 
                                         & \textbf{2} & 3  & 1235 & 411 & 0.522  & 0.0273 & 0.0031 & 0.0045 & 0.0349 \\ \cline{2-10} 
                                         & \textbf{3} & 3  & 1219 & 395 & 0.4587 & 0.0323 & 0.0029 & 0.0061 & 0.0413 \\ \cline{2-10} 
                                         & \textbf{4} & 3  & 1210 & 386 & 0.697  & 0.0273 & 0.0031 & 0.0055 & 0.0359 \\ \hline
\multirow{5}{*}{\textit{Labyrinth Dual}} & \textbf{0} & 3  & 1085 & 369 & 0.4503 & 0.0303 & 0.0029 & 0.0043 & 0.0376 \\ \cline{2-10} 
                                         & \textbf{1} & 2  & 1069 & 380 & 0.3847 & 0.0303 & 0.0027 & 0.0045 & 0.0376 \\ \cline{2-10} 
                                         & \textbf{2} & 3  & 1073 & 378 & 0.4107 & 0.0287 & 0.0024 & 0.0029 & 0,0339 \\ \cline{2-10} 
                                         & \textbf{3} & 3  & 1083 & 376 & 0.41   & 0.0297 & 0.0031 & 0.0052 & 0.038  \\ \cline{2-10} 
                                         & \textbf{4} & 2  & 1079 & 363 & 0.5623 & 0.029  & 0.0029 & 0.0062 & 0.0381 \\ \hline
\end{tabular}%
}
\caption{Resultados de la experimentación.}
\label{tab:exp-results}
\end{table}
\end{landscape}

%%%%%%%%%%%%%%%%%%%%%%%%%%%%%%%%%%%%%%%%%%%%%%%%%%%%%%%%%%%%%%%%%%%%%%%%%%%%%%%%
%                       Capitulo 9: Conclusiones                               %
%%%%%%%%%%%%%%%%%%%%%%%%%%%%%%%%%%%%%%%%%%%%%%%%%%%%%%%%%%%%%%%%%%%%%%%%%%%%%%%%

\chapter{Conclusiones}


\section{Trabajos futuros}

\newpage

% Incluir la bibliografia completa
\nocite{*}
\printbibliography[heading=bibintoc]

\end{document}

